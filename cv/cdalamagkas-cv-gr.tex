%!TEX TS-program = xelatex
\documentclass{mycv}

\begin{document}
	\thispagestyle{plain}
	\begin{minipage}{.7\textwidth}
		\begin{flushleft}
			\name{Χρήστος Δαλαμάγκας}{Βοηθος Ερευνας}{Μηχανικος Δικτυων}
			\contact{(+30) 698 316 0295}{cdalamagkas@gmail.com}{chris.dal}{https://christos.pw}{linkedin.com/in/cdalamagkas}{cdalamagkas}
			\birth{3 Νοεμβρίου 1994}
		\end{flushleft}
	\end{minipage}
	\begin{minipage}{.3\textwidth}
		\begin{flushright}
			\includegraphics[scale=0.2]{photo.jpg}
		\end{flushright}
	\end{minipage}
	%
	\vspace*{-0.75cm}
	%
	\section{Εκπαιδευση}

	\begin{EntryDated}{Πανεπιστήμιο Δυτικής Μακεδονίας}{http://ece.uowm.gr}{2019 -- Τώρα}{Υποψηφιος Διδακτορας, Τμημα Ηλεκτρολογων Μηχανικων και Μηχανικων Υπολογιστων}{1.75cm}
	\begin{Itemize}
		%\item Recognized as integrated master degree (level 7 of EFQ) under government gazette 3987/14-9-2018
		%\item Thesis title: "\textit{Design of Market Mechanism for Dynamic Bandwidth Allocation on XG-PON}".
		%\item Thesis available online: \url{https://christos.pw/thesis}.
		\item Οπτικά Δίκτυα και Βελτιστοποίηση.
		\item Βοηθός έρευνας (πρόγραμμα SPEAR)
	\end{Itemize}
	\end{EntryDated}
	
	\vspace*{0.5cm}
	
	\begin{EntryDatedLogo}{Πανεπιστήμιο Δυτικής Μακεδονίας}{http://ece.uowm.gr}{2012 -- 2017}{Διπλωμα (Πενταετους φοιτησης), Πρ. Τμημα Μηχανικων Πληροφορικης και Τηλεπικοινωνιων}{-3.25cm}{uowm}{0.6}
		\begin{Itemize}
			%\item Recognized as integrated master degree (level 7 of EFQ) under government gazette 3987/14-9-2018
			%\item Thesis title: "\textit{Design of Market Mechanism for Dynamic Bandwidth Allocation on XG-PON}".
			\item Θεωρία παιγνίων στα δίκτυα XG-PON: \url{https://christos.pw/thesis}.
			\item Βαθμός αποφοίτησης: 8.2/10.
		\end{Itemize}
	\end{EntryDatedLogo}
	
	\section{Επαγγελματικη εμπειρια}
		\begin{EntryDatedLogo}{Δημόσια Επιχείρηση Ηλεκτρισμού}{https://www.dei.gr/en}{Μάϊ. 2018 - Τώρα}{Βοηθος Ερευνας}{-1.25cm}{dei}{0.6}
		\begin{Itemize}
			\item Ερευνητής σε προγράμματα Ορίζοντα 2020 (SPEAR, SDN-microSENSE).
			\item Έξυπνα δίκτυα ηλεκτρικής ενέργειας, βιομηχανικά δίκτυα και κυβερνοασφάλεια.
		\end{Itemize}
	\end{EntryDatedLogo}
	
	\vspace*{0.5cm}
	
	\begin{EntryDatedLogo}{ΙΙΕΚ ΑΛΦΑ}{https://iekalfa.gr}{Οκτ. 2018 - Τώρα}{Δασκαλος}{-1cm}{alfa}{0.6}
		\begin{Itemize}
			\item Δίκτυα υπολογιστών και τηλεπικοινωνίες.
			\item Λειτουργικά Συστήματα και Αντικειμενοστραφής Προγραμματισμός (C++).
		\end{Itemize}
	\end{EntryDatedLogo}

	\vspace*{0.5cm}
		
	\begin{EntryDatedLogo}{Πανεπιστήμιο του Brighton}{https://www.brighton.ac.uk}{Ιουλ. - Σεπτ. 2017}{Βοηθος Ερευνας}{-0.45cm}{brighton}{0.6}
	%	\begin{Itemize}
	%		\item Research on XG-PON and NG-PON2 networks.
	%		\item Development of NG-PON2 simulator in MATLAB.
	%		\item Worked on a novel MAC algorithm for NG-PON2.	
%	\end{Itemize}
	\end{EntryDatedLogo}

	\vspace*{0.5cm}	

	\begin{EntryDatedLogo}{Πανεπιστήμιο Δυτικής Μακεδονίας}{http://icte.uowm.gr}{Μάρ. 2017 - Τώρα}{Βοηθος εργαστηριων}{-1cm}{uowm}{0.6}
	\begin{Itemize}
		\item Δίκτυα Υπολογιστών, Ασφάλεια Υπολογιστών και Δικτύων.
		\item Γεγονοστραφής προσομοίωση.
	\end{Itemize}
	\end{EntryDatedLogo}

	\vspace*{0.75cm}	

	\begin{EntryDatedLogo}{IntelliSolutions S.A}{http://intelli-corp.com}{Ιούλ. - Αυγ. 2016}{Μηχανικος Δικτυων και Συστηματων (Ασκουμενος)}{-0.4cm}{intelli}{0.75}
	%\begin{Itemize}
	%	\item Design and implementation of new enterprise network.
	%	\item Network security planning and implementation (pfSense, OpenVPN, firewall, squid).
	%	\item Windows Server, Active Directory and SQL failover cluster setup, type 2 virtualization (VMware).
	%\end{Itemize}
	\end{EntryDatedLogo}
	\newpage
	\section{Δεξιοτητες}
	\begin{tabular}{m{4.5cm} m{13cm}}\renewcommand{\arraystretch}{2}
		\textbf{Τηλεπικοινωνίες}   		& ITU-T PONs, Traffic Engineering, Simulation (OmNET++). \\
		\textbf{Δικτύωση}   			& TCP/IP, OpenVSwitch, Cisco IOS, RouterOS, ArubaOS, Routing and Switching.\\
		\textbf{Ασφάλεια}				& TLS, PKI, OpenVPN, pfSense, ACLs. \\
		\textbf{Διαχείριση Συστημάτων}	& Unix, LEMP stack, Windows Server, Active Directory. \\
		\textbf{Εικονικοποίηση}			& Xen (XCP-ng), ESXi, Proxmox, VMware/Virtualbox, Docker.\\ 
		\textbf{Προγραμματισμός} 	    & Java, MATLAB, Python, C/C++, Basics of web development. \\
		\textbf{Διάφορα}				& Troubleshooting, Project Management, Office Suite, MS Visio, \LaTeX. \\
		\textbf{Γλώσσες} 				& Greek (native), English (C2), German (C1). 
	\end{tabular}
	
	\section{Επιλεγμενες Δημοσιευσεις}
	\pubentry{1}{C. Dalamagkas, P. Sarigiannidis, S. Kapetanakis and I. Moscholios, "Dynamic scheduling in {TWDM}-{PONs} using game theory", \textit{Optical Switching and Networking}, Dec. 2017, DOI: \href{https://doi.org/10.1016/j.osn.2017.12.004}{\texttt{10.1016/j.osn.2017.12.004}}.}{2018}
	
	\pubentry{2}{C. Dalamagkas, P. Sarigiannidis, I. Moscholios, T. Lagkas and M.S. Obaidat, "PAS: A Fair Game-Driven DBA Scheme for XG-PON Systems", \textit{11th International Symposium on Communication Systems, Networks, and Digital Signal Processing}, Jul. 2018. DOI: 
	\href{https://doi.org/10.1109/CSNDSP.2018.8471787}{\texttt{10.1109/CSNDSP.2018.8471787}}}{2018}
	
	\vspace{-0.25cm}
	\section{Επαγγελματικες Πιστοποιησεις}
	\begin{EntryDatedLogo}{Cisco Certified Network Associate (CCNA)}{https://www.cisco.com/}{\scshape{Ιουλ. 2019}}{}{-0.75cm}{cisco}{0.6}
	\end{EntryDatedLogo}
	\vspace{0.25cm}
	
	\section{Εθελοντικες Δραστηριοτητες και Συνδρομες}
	\vspace*{0.125cm}	
	\begin{EntryDatedLogo}{Πανεπιστήμιο Δυτικής Μακεδονίας}{https://uowm.gr}{Μαρτ. 2016 - Τώρα}{Συντακτης εκπαιδευτικου υλικου}{-1.1cm}{uowm}{0.6}
	\begin{Itemize}
		\item Συγγραφή πρωτότυπου εκπαιδευτικού υλικού για το μάθημα «Σχεδίαση Δικτύων».
		\item Η ύλη περιλαμβάνει δρομολόγηση και μεταγωγή με συσκευές Cisco και Mikrotik.
	\end{Itemize}
	\end{EntryDatedLogo}

	\vspace*{0.5cm}
	
	\begin{EntryDatedLogo}{Ινστιτούτο Ηλεκτρολόγων και Ηλεκτρονικών Μηχανικών}{https://www.ieee.org/}{Σεπτ. 2013 -- Τώρα}{Μελος της IEEE Communications Society}{-0.5cm}{ieee}{0.6}
	\end{EntryDatedLogo}

\end{document}
